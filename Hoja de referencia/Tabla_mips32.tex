\documentclass{article}
\usepackage[utf8]{inputenc}
\usepackage[T1]{fontenc}
\usepackage{amsmath}
\usepackage[spanish]{babel}
\usepackage{geometry}
\usepackage{array}
\usepackage{tabularx}
\usepackage{booktabs}
\usepackage[table]{xcolor}
\usepackage{caption}

\geometry{margin=1in}

% Paleta y estilo de tabla
\definecolor{HeaderBg}{HTML}{2F3B52} % azul grisáceo oscuro
\definecolor{HeaderText}{HTML}{FFFFFF}
\definecolor{RowLight}{HTML}{F5F7FA} % gris muy claro
\definecolor{RowDark}{HTML}{E9EEF5}  % gris claro

\newcolumntype{L}[1]{>{\raggedright\arraybackslash}p{#1}}
\newcolumntype{C}[1]{>{\centering\arraybackslash}p{#1}}

\begin{document}

\begin{titlepage}
\centering
\vspace*{2cm}

{\Huge\bfseries Manual de Instrucciones MIPS32 en MARS\par}
\vspace{1cm}
{\Large\itshape Tabla completa de procedimientos más usados\par}

\vfill

{\Large Estudiante: Jose Natera\par}
{\large CI: 31185494\par}


\vspace{1cm}
{\large\textbf{Arquitectura del Computador}\par}

\end{titlepage}



\section*{Tabla de registros MIPS}

\captionsetup{labelfont=bf}
\begin{table}[h!]
\renewcommand{\arraystretch}{1.25}
\rowcolors{2}{RowLight}{RowDark}
\setlength{\tabcolsep}{10pt}
\begin{tabular}{L{0.10\linewidth} C{0.18\linewidth} L{0.50\linewidth}}
\rowcolor{HeaderBg}
\textcolor{HeaderText}{\textbf{Nombre}} &
\textcolor{HeaderText}{\textbf{Código}} &
\textcolor{HeaderText}{\textbf{Uso común}} \\

\$\texttt{zero}      & 0       & Constante 0 \\
\$\texttt{v0} -- \$\texttt{v1} & 2--3     & Valores de retorno de funciones / \texttt{syscall} \\
\$\texttt{a0} -- \$\texttt{a3} & 4--7     & Argumentos para funciones / \texttt{syscall} \\
\$\texttt{t0} -- \$\texttt{t7} & 8--15    & Temporales (no conservados entre llamadas) \\
\$\texttt{s0} -- \$\texttt{s7} & 16--23   & Variables locales (se conservan entre llamadas) \\


\$\texttt{t8} -- \$\texttt{t9} & 24--25   & Temporales adicionales

\\\$\texttt{k0} -- \$\texttt{k1} & 26--27   & Reservados para núcleo del Sistema Operativo

\\\$\texttt{gp}  & 28   & Puntero a global

\\\$\texttt{sp}  & 29   & Puntero a pila

\\\$\texttt{fp}  & 30   & Puntero a marco

\\
\$\texttt{ra}         & 31      & Dirección de retorno (\texttt{jal}, \texttt{jalr}) \\
\end{tabular}
\end{table}



\section*{Formato de instrucciones del MIPS32}

\begin{table}[h!]
\renewcommand{\arraystretch}{1.25}
\rowcolors{2}{RowLight}{RowDark}
\setlength{\tabcolsep}{2pt}
\begin{tabular}{L{0.07\linewidth} C{0.12\linewidth} C{0.08\linewidth} C{0.08\linewidth} C{0.08\linewidth} C{0.13\linewidth} C{0.13\linewidth} L{0.22\linewidth}}
\rowcolor{HeaderBg}
\textcolor{HeaderText}{\textbf{Tipo}} &
\textcolor{HeaderText}{\textbf{Cod oper}} &
\textcolor{HeaderText}{\textbf{rs}} &
\textcolor{HeaderText}{\textbf{rt}} &
\textcolor{HeaderText}{\textbf{rd}} &
\textcolor{HeaderText}{\textbf{desplaz}} &
\textcolor{HeaderText}{\textbf{func}} &
\textcolor{HeaderText}{\textbf{Instrucción ejemplo}} \\

R & xxxxxx & \$\texttt{s1} & \$\texttt{s2} & \$\texttt{s0} & 00000 & 100000 & \texttt{add \$s0, \$s1, \$s2} \\
I & xxxxxx & \$\texttt{s1} & \$\texttt{s0} & — & — & — & \texttt{addi \$s0, \$s1, 5} \\
J & xxxxxx & — & — & — & — & — & \texttt{j label} \\
\rowcolor{RowLight}
\rowcolor{RowLight} &
31-26 & 25-21 & 20-16 & 15-11 & 10-6 & 5-0 & \\

\end{tabular}
\end{table}


\section*{}
\begin{table}[h!]
\renewcommand{\arraystretch}{1.25}
\rowcolors{3}{RowLight}{RowDark}
\setlength{\tabcolsep}{3pt}
\begin{tabular}{L{0.27\linewidth} C{0.17\linewidth} L{0.30\linewidth} L{0.25\linewidth}}
\rowcolor{HeaderBg}
\textcolor{HeaderText}{\textbf{Nombre}} &
\textcolor{HeaderText}{\textbf{Nemotécnica}} &
\textcolor{HeaderText}{\textbf{Operación}} &
\textcolor{HeaderText}{\textbf{Ejemplo}} \\

\rowcolor{white}
\multicolumn{4}{c}{\textbf{Operaciones Booleanas}} \\

And              & \texttt{and}   & \(\text{rd} = \text{rs} \land \text{rt}\)         & \texttt{and \$t0, \$t1, \$t2} \\
And inmediato    & \texttt{andi}  & \(\text{rd} = \text{rs} \land \text{imm}\)        & \texttt{andi \$t0, \$t1, 0x00FF} \\
Or               & \texttt{or}    & \(\text{rd} = \text{rs} \lor \text{rt}\)          & \texttt{or \$t0, \$t1, \$t2} \\
Or inmediato     & \texttt{ori}   & \(\text{rd} = \text{rs} \lor \text{imm}\)         & \texttt{ori \$t0, \$t1, 0x0100} \\
Xor              & \texttt{xor}   & \(\text{rd} = \text{rs} \oplus \text{rt}\)        & \texttt{xor \$t0, \$t1, \$t2} \\
Xor inmediato    & \texttt{xori}  & \(\text{rd} = \text{rs} \oplus \text{imm}\)       & \texttt{xori \$t0, \$t1, 0x000F} \\
Nor              & \texttt{nor}   & \(\text{rd} = \lnot(\text{rs} \lor \text{rt})\)   & \texttt{nor \$t0, \$t1, \$t2} \\

\rowcolor{white}
\multicolumn{4}{c}{\textbf{Operaciones de comparación}} \\

Menor  que                    & \texttt{slt}   & \(\text{rd} = (\text{rs} < \text{rt})\)         & \texttt{slt \$t0, \$t1, \$t2} \\
Menor que inmediato            & \texttt{slti}  & \(\text{rd} = (\text{rs} < \text{imm})\)        & \texttt{slti \$t0, \$t1, 10} \\
Menor que (sin signo)          & \texttt{sltu}  & \(\text{rd} = (\text{rs} < \text{rt})\)         & \texttt{sltu \$t0, \$t1, \$t2} \\


Igual que             & \texttt{seq}   & \(\text{rd} = (\text{rs} == \text{rt})\)        & \texttt{seq \$t0, \$t1, \$t2} \\
Distinto que           & \texttt{sne}   & \(\text{rd} = (\text{rs} \neq \text{rt})\)      & \texttt{sne \$t0, \$t1, \$t2} \\
Mayor que             & \texttt{sgt}   & \(\text{rd} = (\text{rs} > \text{rt})\)         & \texttt{sgt \$t0, \$t1, \$t2} \\
Mayor o igual  que    & \texttt{sge}   & \(\text{rd} = (\text{rs} \geq \text{rt})\)      & \texttt{sge \$t0, \$t1, \$t2} \\
Menor o igual que     & \texttt{sle}   & \(\text{rd} = (\text{rs} \leq \text{rt})\)      & \texttt{sle \$t0, \$t1, \$t2} \\


\rowcolor{white}
\multicolumn{4}{c}{\textbf{Saltos incondicionales}} \\

Salto absoluto                  & \texttt{j}     & \(\text{PC} = \text{label}\)                 & \texttt{j label} \\
Salto y enlace                  & \texttt{jal}   & \(\text{ra} = \text{PC} + 4;\ \text{PC} = \text{label}\) & \texttt{jal label} \\
Salto a dirección en registro   & \texttt{jr}    & \(\text{PC} = \text{ra}\)                        & \texttt{jr \$ra} \\
Salto y enlace por registro     & \texttt{jalr}  & \(\text{ra} = \text{PC} + 4;\ \text{PC} = \text{rs}\) & \texttt{jalr \$t0} \\

\rowcolor{white}
\multicolumn{4}{c}{\textbf{Saltos condicionales}} \\

Igualdad               & \texttt{beq}   & si \(\text{rs} == \text{rt}\), ir a \texttt{label}   & \texttt{beq \$t0, \$t1, label} \\
Desigualdad            & \texttt{bne}   & si \(\text{rs} \neq \text{rt}\), ir a \texttt{label} & \texttt{bne \$t0, \$t1, label} \\
Menor que cero         & \texttt{bltz}  & si \(\text{rs} < 0\), ir a \texttt{label}            & \texttt{bltz \$t0, label} \\
Mayor o igual que cero & \texttt{bgez}  & si \(\text{rs} \geq 0\), ir a \texttt{label}          & \texttt{bgez \$t0, label} \\
Mayor que cero         & \texttt{bgtz}  & si \(\text{rs} > 0\), ir a \texttt{label}             & \texttt{bgtz \$t0, label} \\
Menor o igual que cero & \texttt{blez}  & si \(\text{rs} \leq 0\), ir a \texttt{label}          & \texttt{blez \$t0, label} \\


\rowcolor{white}
\multicolumn{4}{c}{\textbf{Desplazamientos lógicos y rotacion a izquierda}} \\

Desplazamiento lógico izquierda (inmediato) & \texttt{sll}   & \(\text{rd} = \text{rt} \ll \text{shamt}\) & \texttt{sll \$t0, \$t1, 2} \\
Desplazamiento lógico izquierda (por registro) & \texttt{sllv}  & \(\text{rd} = \text{rt} \ll \text{rs}\) & \texttt{sllv \$t0, \$t1, \$t2} \\

\rowcolor{white}
\multicolumn{4}{c}{\textbf{Desplazamientos lógicos y rotacion a derecha}} \\

Desplazamiento lógico derecha (inmediato) & \texttt{srl}   & \(\text{rd} = \text{rt} \gg \text{shamt}\) & \texttt{srl \$t0, \$t1, 2} \\
Desplazamiento lógico derecha (por registro) & \texttt{srlv}  & \(\text{rd} = \text{rt} \gg \text{rs}\) & \texttt{srlv \$t0, \$t1, \$t2}

\end{tabular}
\end{table}



\section*{}
\begin{table}[!ht]
\renewcommand{\arraystretch}{1.25}
\rowcolors{3}{RowLight}{RowDark}
\setlength{\tabcolsep}{3pt}
\begin{tabular}{L{0.27\linewidth} C{0.17\linewidth} L{0.30\linewidth} L{0.25\linewidth}}
\rowcolor{HeaderBg}
\textcolor{HeaderText}{\textbf{Nombre}} &
\textcolor{HeaderText}{\textbf{Nemotécnica}} &
\textcolor{HeaderText}{\textbf{Operación}} &
\textcolor{HeaderText}{\textbf{Ejemplo}} \\


\rowcolor{white}
\multicolumn{4}{c}{\textbf{Desplazamientos aritméticos a derecha}} \\

Desplazamiento aritmético derecha (inmediato) & \texttt{sra}   & \(\text{rd} = \text{rt} \gg \text{shamt}\) & \texttt{sra \$t0, \$t1, 2} \\
Desplazamiento aritmético derecha (por registro) & \texttt{srav}  & \(\text{rd} = \text{rt} \gg \text{rs}\)  & \texttt{srav \$t0, \$t1, \$t2} \\
Desplazamiento aritmético derecha (pseudo)      & \texttt{srar}  & \(\text{rd} = \text{rt} \gg \text{imm}\)  & \texttt{srar \$t0, \$t1, 2} \\



\rowcolor{white}
\multicolumn{4}{c}{\textbf{Acceso a memoria}} \\

Mover desde LO         & \texttt{mflo}  & \(\text{rd} = \text{LO}\)                             & \texttt{mflo \$t0} \\
Mover desde HI         & \texttt{mfhi}  & \(\text{rd} = \text{HI}\)                             & \texttt{mfhi \$t0}\\
Cargar byte        & \texttt{lb}   & \(\text{rd} = \text{mem}[\text{rs} + \text{offset}]\) & \texttt{lb \$t0, 0(\$a0)} \\

Guardar byte                   & \texttt{sb}   & \(\text{mem}[\text{rs} + \text{offset}] = \text{rt}\) & \texttt{sb \$t0, 0(\$a0)} \\
Cargar halfword    & \texttt{lh}   & \(\text{rd} = \text{mem}[\text{rs} + \text{offset}]\) & \texttt{lh \$t0, 2(\$a0)} \\

Guardar halfword               & \texttt{sh}   & \(\text{mem}[\text{rs} + \text{offset}] = \text{rt}\) & \texttt{sh \$t0, 2(\$a0)} \\
Cargar palabra                 & \texttt{lw}   & \(\text{rd} = \text{mem}[\text{rs} + \text{offset}]\) & \texttt{lw \$t0, 4(\$gp)} \\
Guardar palabra                & \texttt{sw}   & \(\text{mem}[\text{rs} + \text{offset}] = \text{rt}\) & \texttt{sw \$t0, 8(\$gp)} \\

\rowcolor{white}
\multicolumn{4}{c}{\textbf{Transferencia de datos}} \\

Cargar inmediato alto          & \texttt{lui}  & \(\text{rd} = \text{imm} \ll 16\) & \texttt{lui \$t0, 0x1234} \\
Cargar inmediato (pseudo)               & \texttt{li}   & \(\text{rd} = \text{imm}\)  & \texttt{li \$t0, 42} \\
Cargar dirección  (pseudo)             & \texttt{la}   & \(\text{rd} = \text{label}\)  & \texttt{la \$t0, label} \\
Mover entre registros (pseudo)          & \texttt{move} & \(\text{rd} = \text{rs}\)  & \texttt{move \$t0, \$t1}

\\
\rowcolor{white}
\multicolumn{4}{c}{\textbf{Operaciones aritméticas}} \\

Suma con signo & \texttt{add}  & \(\text{rd} = \text{rs} + \text{rt}\)        & \texttt{add \$t0, \$t1, \$t2} \\
Suma inmediata & \texttt{addi} & \(\text{rd} = \text{rs} + \text{imm}\)     & \texttt{addi \$t0, \$t1, 10} \\
Suma sin signo & \texttt{addu} & \(\text{rd} = \text{rs} + \text{rt}\)      & \texttt{addu \$t0, \$t1, \$t2} \\
Resta con signo & \texttt{sub}  & \(\text{rd} = \text{rs} - \text{rt}\)      & \texttt{sub \$t0, \$t1, \$t2} \\
Resta sin signo & \texttt{subu} & \(\text{rd} = \text{rs} - \text{rt}\)     & \texttt{subu \$t0, \$t1, \$t2} \\
Multiplicación entera & \texttt{mult}  & \(\text{LO} = \text{rs} \cdot \text{rt}\) & \texttt{mult \$t1, \$t2} \\
Multiplicación sin signo & \texttt{multu} & \(\text{LO} = \text{rs} \cdot \text{rt}\) & \texttt{multu \$t1, \$t2} \\
División & \texttt{div}  & \(\text{LO} = \lfloor\text{rs}/\text{rt}\rfloor;\ \text{HI} = \text{rs}\text{ mod }
\text{rt}\) & \texttt{div \$t1, \$t2} \\
División sin signo & \texttt{divu} & \(\text{LO} = \lfloor\text{rs}/\text{rt}\rfloor;\ \text{HI} = \text{rs}\text{ mod }\text{rt}\) & \texttt{divu \$t1, \$t2} \\
Valor absoluto & \texttt{abs}  & \(\text{rd} = \lvert \text{rs} \rvert\)    & \texttt{abs \$t1} \\
Negación & \texttt{neg}  & \(\text{rd} = -\text{rs}\)                 & \texttt{neg \$t1, \$t2}


\end{tabular}
\end{table}

\end{document}

