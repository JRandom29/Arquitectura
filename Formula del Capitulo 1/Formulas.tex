\documentclass{article}
\usepackage[landscape]{geometry}
\usepackage{url}
\usepackage{multicol}
\usepackage{amsmath}
\usepackage{esint}
\usepackage{amsfonts}
\usepackage{tikz}

\usetikzlibrary{decorations.pathmorphing}
\usepackage{amsmath,amssymb}

\usepackage{colortbl}
\usepackage{xcolor}
\usepackage{mathtools}
\usepackage{amsmath,amssymb}
\usepackage{enumitem}
\makeatletter

\newcommand*\bigcdot{\mathpalette\bigcdot@{.5}}
\newcommand*\bigcdot@[2]{\mathbin{\vcenter{\hbox{\scalebox{#2}{$\m@th#1\bullet$}}}}}
\makeatother

\title{Formulario Arquitectura}
\usepackage[brazilian]{babel}
\usepackage[utf8]{inputenc}

\advance\topmargin-.8in
\advance\textheight3in
\advance\textwidth3in
\advance\oddsidemargin-1.45in
\advance\evensidemargin-1.45in
\parindent0pt
\parskip2pt
\newcommand{\hr}{\centerline{\rule{3.5in}{1pt}}}

\begin{document}


\begin{center}{\huge{\textbf{Formulario del capitulo 1}}}\\
\end{center}
\begin{multicols*}{2}

\tikzstyle{mybox} = [draw=black, fill=white, very thick,
    rectangle, rounded corners, inner sep=10pt, inner ysep=10pt]
\tikzstyle{fancytitle} =[fill=black, text=white, font=\bfseries]

%--------------- prestaciones -----------%
\begin{tikzpicture}
\node [mybox] (box){
    \begin{minipage}{0.46\textwidth}
        
        \begin{tabular}{lp{8cm} l} 
            $Prestaciones_{x} = \displaystyle\frac{1}{\text{Tiempo de ejecucion}_{x}}$
            \\\\
            a menudo se desea relacionar cuantitativamante las prestaciones de dos
            \\
            máquinas diferentes.
            \\\\
            
            $n = \displaystyle\frac{\text{Prestaciones}_x}{\text{Presta ciones}_{y}}$
            \\\\
            Si X es n veces más rápida que Y, entonces el tiempo de ejecución de Y 
            \\
            es n veces mayor que el de X:
            \\\\
            $ \displaystyle\frac{\text{Prestaciones}_x}{\text{Presta ciones}_{y}} = \displaystyle\frac{\text{tiempo de ejecucion}_y}{\text{Tiempo de ejecucion}_x} = n$ 
	    \end{tabular}	
    \end{minipage}
};
%------------Titulo superior Derecho ---------------------
\node[fancytitle, right=10pt] at (box.north west) 
{Prestaciones};
\end{tikzpicture}


%--------------- Prestaciones de la CPU -----------%
\begin{tikzpicture}
\node [mybox] (box){
    \begin{minipage}{0.46\textwidth}
    
	\begin{tabular}{lp{8cm} l}
    
    $
    \begin{gathered}
        \text{Tiempo de ejecución de} 
        \\[-1ex]
        \text{CPU para un programa}
    \end{gathered}
    
    \text{ = }
    \begin{gathered}
        \text{Ciclos de reloj de la} 
        \\[-1ex]
        \text{CPU para el programa}
    \end{gathered}
    \times
    \begin{gathered}
        \text{Tiempo del}
        \\[-1ex]
        \text{ciclo del reloj}
    \end{gathered}
    $
    \\\\
    \text{Alternativamente, ya que la frecuencia de reloj es la inversa del tiempo de ciclo}
    \\\\
    $
    \begin{gathered}
        \text{Tiempo de ejecución de} 
        \\[-1ex]
        \text{CPU para un programa}
    \end{gathered}
    \text{ = }

    \displaystyle\frac{\text{Ciclos de reloj de la CPU para el programa}}{\text{Frecuencia de reloj }}    
    $
    \\\\
     $\blacktriangle$
     \textbf{Prestaciones de las instrucciones}
    \\\\
    $
    \text{Ciclos de reloj de CPU }
    \text{ = }
    
    \begin{gathered}
        \text{Instrucciones} \\[-1ex]
        \text{de un programa}
    \end{gathered}
    \times
    
    \begin{gathered}
        \text{ Media de ciclos} \\[-1ex]
        \text{ por instruccion}
    \end{gathered}
    $

    \\\\
     $\blacktriangle$
     \textbf{ecuación clásica de las prestaciones de la CPU}
    \\\\
    $
    \begin{gathered}
        \text{Tiempo}\\[-1ex]
        \text{de ejecucion}
    \end{gathered}
    
    \text{ = }
    
    \begin{gathered}
        \text{Número}\\[-1ex]
        \text{de instrucciones}
    \end{gathered}
    
    \times
    
    \begin{gathered}
        \text{Ciclos de reloj }\\[-1ex]
        \text{por instrucción (CPI)}
    \end{gathered}
    
    \times
    \text{Tiempo de ciclo}
    $
    \\\\
    \text{o bien, dado que la frecuencia es el inverso del tiempo de ciclo:}
    \\\\
  $
    \begin{gathered}
        \text{Tiempo}\\[-1ex]
        \text{de ejecucion}
    \end{gathered}
    \text{ = }
    \displaystyle\frac{\text{Numero de instrucciones}\times\text{CPI}}{\text{Frecuencia de reloj}}
    $
    \end{tabular}
    \end{minipage}
};
%------------Titulo superior ---------------------
\node[fancytitle, right=10pt] at (box.north west) 
{Prestaciones de la CPU};
\end{tikzpicture}


%--------------- Potencia -----------%
\begin{tikzpicture}
\node [mybox] (box){
    \begin{minipage}{0.46\textwidth}
    
	\begin{tabular}{lp{8cm} l}
    
    $ \text{Potencia = carga capacitiva} \times voltaje^2 \times \text{ frecuencia de conmutación}$
    \\\\
    $\blacktriangle$
    \textbf{Coste de un circuito integrado}
    \\\\
    \text{El coste de un circuito integrado se puede expresar con tres ecuaciones simples:}
    \\\\
    $
    \begin{gathered}
        \text{coste}\\[-1ex]
        \text{por dado}
    \end{gathered}
    \text{ = }
    
    \displaystyle\frac{\text{coste por oblea}}{\begin{gathered}\text{dado por oblea}\end{gathered} \times \text{factor de producción}}
    $
\\[1.5em] 
$
\begin{gathered}
    \text{dados}\\[-1ex]
    \text{por oblea}
\end{gathered}
\text{ = }
\displaystyle\frac{\text{área de la oblea}}{\text{área del dado}}
$
\\[1.5em]
$
\begin{gathered}
    \text{factor}\\[-1ex]
    \text{de producción}
\end{gathered}
\text{ = }
\displaystyle\frac{1}{\left(1 + \left(\text{defectos por área} \times \text{área del dado}/2\right)\right)^{2}}
    $

    
    \\\\
    $\blacktriangle$
    \textbf{Formula de la media geometrica}
    \\\\
    $
    \sqrt[n]{\prod_{i=1}^{n} \text{Relaciones de tiempos de ejecución}_i}
    $
    \end{tabular}
    \end{minipage}
};
%------------Titulo superior ---------------------
\node[fancytitle, right=10pt] at (box.north west) 
{Potencia relativa};
\end{tikzpicture}

\begin{tikzpicture}
\node [mybox] (box){
    \begin{minipage}{0.46\textwidth}
    
	\begin{tabular}{lp{8cm} l}
    
    
    regla que establece que el aumento posible de las prestaciones
    con una mejora
    \\ determinada está limitado por la cantidad en que se usa la mejora.
    \\\\
    $
    \begin{gathered}
    \text{tiempo de ejecución}\\[-1ex]
    \text{después de las mejoras}
\end{gathered}
\text{ = }
\displaystyle\frac{\begin{gathered}\text{tiempo de ejecución}\\[-1ex]\text{por la mejora}\end{gathered}}{\begin{gathered}\text{cantidad}\\[-1ex]\text{de mejora}\end{gathered}}
\text{ + }
\begin{gathered}
    \text{tiempo de ejecución}\\[-1ex]
    \text{no afectado}
\end{gathered}
    $
    \\\\
    $\blacktriangle$
    Una alternativa al tiempo de ejecución son los MIPS (millones de \\instrucciones por segundo). Para un programa dado\\\\
    $\text{MIPS} = \displaystyle\frac{\text{número de instrucciones}}{\text{tiempo de ejecución} \times 10^6}$
    \\\\
    $\text{MIPS} = \displaystyle\frac{\text{frecuencia de reloj}}{\text{CPI} \times 10^6}$    
    
    \end{tabular}
    \end{minipage}
};
%------------Titulo superior ---------------------
\node[fancytitle, right=10pt] at (box.north west) 
{Ley de Amdhal};
\end{tikzpicture}


\end{multicols*}
\end{document}

